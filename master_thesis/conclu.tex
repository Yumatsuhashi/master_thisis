%#!make thesis.dvi
%======================================================================
% 結論
%======================================================================
\chapter{結論}
{}
\label{chap:conclu}
\section{本研究の主たる成果}

本論文は,誘導電力伝送システムの漏洩磁界の低減を目的として,シングルターンコイルとキャパシタのみから成る漏洩磁界キャンセラを提案しその評価を行った.本研究の成果は以下の2点である.

1点目は,提案手法が送受電器の形状や大きさに非依存な漏洩磁界低減手法であることを示した点である.本研究ではループコイルの磁気双極子モーメントに基づく電磁界解析を通して,送受電器の構造,寸法,位置に関するパラメータを一切含まない設計式を理論的に得た.また,シミュレーションと実験を用いて提案するキャンセルコイルの評価を行い,わずかな電力伝送効率の低下で漏洩磁界の低減効果を確認し,複数受電器のような構造の磁界結合型無線電力伝送システムにも本手法が有効なことを確認した.

2点目は,誘導型無線電力伝送システムにおける受電器の位置ずれに対して,提案手法が適応可能であることを示した点である.本手法の原理では送電器と受電器の軸が同一である前提のもとで議論をすすめて提案する設計式を得たが,シミュレーションと実装を通して本手法が位置ずれが起きている場合に対しても,わずかな電力伝送効率の低下で漏洩磁界の低減効果があることを確認した.

本手法は,既存のリアクティブシールドなどの漏洩磁界低減手法と併用することが可能であり,そうすることで一層の漏洩磁界低減効果が得られることが期待される.漏洩磁界が抑制されることにより,無線電力伝送システムで扱える電力が大きくなるため,無線電力伝送が使われる応用先の幅を広くすることができる.

\section{今後の課題}

本研究における今後の課題と展望を述べる.

本研究では,キャンセルコイルのパラメータとして,その半径と導線の太さを変化させた場合に漏洩磁界の低減効果がどう変化するかを評価し,提案手法の有効性を確認した.その際に送受電器の大きさを一定にした状態で評価を行ったが,キャンセルコイルの漏洩磁界低減効果が最も大きくなるような条件は送受電器の大きさによって変化する.また,導線の種類の評価として導線の太さに関する評価を行ったが,その導線の材質などによっても漏洩磁界の低減効果には変化があることが予想される.そのため,送受電システムの大きさなどに応じた最適なキャンセルコイルの設計条件の検証が今後の課題としてあげられる.

そして,本研究では受電器位置ずれに対する実装評価の際に,受電電力を一定にした状態で実験を行うと,受電器が位置ずれを起こしている場合の漏洩磁界は位置ずれがない場合に比べて漏洩磁界は大きくなることを前提にした評価を行った.具体的には,キャンセルコイルが存在し受電器位置ずれがある場合と,キャンセルコイルがなく受電器位置ずれがない場合における磁界強度の測定結果との差を評価に用いた.しかし,位置ずれに対する評価を適切に行うにはキャンセルコイルがなく位置ずれがある場合の評価が必要である.

また,本研究ではキャンセルコイルの導線の太さを変化させた評価と複数の受電器をもつ電力伝送システムへの評価をシミュレーションにて検証したが,実装において検証を行っていない.そのため,実際に実装し評価を行い,理論やシミュレーションとの差異について検証と考察を行う必要がある.