%#!make thesis.dvi
%======================================================================
% 結論
%======================================================================
\chapter{結論}
{}
\label{chap:conclu}
\section{本研究の主たる成果}

本論文は,参照信号削減下における時系列CSI予測モデルの重み更新を目的として,時系列補間型学習データを用いたチャネル予測機構を提案しその評価を行った.本研究の成果は以下の2点である.

1点目は,チャネル予測モデルの重み更新に有効な補間手法を特定した点である.線形補間,スプライン補間,多項式近似,ニューラルネットワーク補間の4種類を比較した結果,スプライン補間と4次多項式近似がNMSE 0.01以下の高精度を達成した.パイロット信号の送信頻度を低減し欠損率が25\%から50\%に増加した場合においても,両手法は安定した補間精度を維持した.この結果から,時系列補間型学習データの生成に適した補間手法として,スプライン補間および4次多項式近似が有効であることを示した.

2点目は,参照信号削減下でもモデル更新が可能な枠組みを提案し,その有効性を確認した点である.予測スロットでは参照信号を送信しないため正解ラベルが欠損するという問題に対し,後続スロットで得られる推定値から補間により欠損時刻のCSIを事後的に算出し,推定値と補間値で構成される学習サンプルを蓄積してモデルを更新する手法を提案した.レイトレーシングにより生成したCSI系列を用いた評価では,新基地局設置時に約79\%,セル内チャネル統計変動時に約73\%から74\%の予測精度改善を達成した.パイロット削減率を50\%に拡大した場合においても,ファインチューニング後のモデルは予測を用いない場合を上回る性能を維持した.

以上の成果により,参照信号削減を維持しながら環境変化に追従したモデル更新が実現可能であることを示した.提案機構は,新規基地局への展開およびセル内の電波伝搬環境変化の両方に対応でき,時系列CSI予測の実運用における適応性向上に寄与する.

\section{今後の課題}

本研究における今後の課題と展望を述べる.

第一に,評価データの多様性に関する課題がある.本研究では錦糸町および八重洲の地形データから生成したCSI系列を用いて評価したが,より多くの地域や環境条件を対象とした検証が求められる.都市部以外の郊外や屋内環境など,異なる電波伝搬特性を持つ環境での有効性を確認する必要がある.

第二に,補間手法の拡張に関する課題がある.本研究ではスプライン補間や多項式近似といった解析的手法と単純なニューラルネットワークを比較したが,GANや拡散モデルなどより高度な生成モデルを用いた補間手法についても検討の余地がある.パイロット削減割合の拡大に伴い予測改善率が低下する傾向が確認されたため,高削減率においても精度を維持できる補間手法の探索が必要である.

第三に,ベースラインモデルの強化に関する課題がある.本研究ではMLPおよびLSTMを用いたが,ファインチューニングを必要としない汎化性能の高いモデルとの比較により,提案機構の位置づけをより明確にできる.

第四に,シミュレーション条件の現実性に関する課題がある.本研究では単一端末の歩行軌跡を対象としたが,端末間干渉の考慮,新幹線やバスなど高速移動体への対応,場所ごとのデータ分布の偏りを考慮したサンプリング手法の検討が求められる.

第五に,評価指標の拡張に関する課題がある.本研究ではNMSEを用いてCSI予測精度を評価したが,予測したCSIを用いたビームフォーミングの結果として得られるスループットや誤り率など,通信品質に直結する指標での評価が実用性の観点から重要である.
