%#!make thesis.dvi
%======================================================================
% Publications
%======================================================================
\chapter*{\publicationname}
\addcontentsline{toc}{chapter}{\publicationname}


%----------------------------------------------------------------------
\section*{論文誌(査読あり) }
\begin{publication}{99}
\bibitem{itu_journal}
Ilan Correa, Ailton Oliveira, Bojian Du, Cleverson Nahum, Daisuke Kobuchi, Felipe Bastos, \underline{Hirofumi Ohzeki},
João Borges, Mohit Mehta, Ryoma Kondo, Sundesh Gupta, and Aldebaro Klautau, ''Simultaneous beam selection and users scheduling
evaluation in a virtual world with reinforcement learning,'' ITU Journal on Future and Evolving Technologies (ITU J-FET). (投稿予定)

%%-------書き方参考------%
% \bibitem{jaxa}
% \underline{小渕大輔}, 薮田直人, 成末義哲, 中岡俊裕, 川崎繁男, ``アクティブ集積アンテナ用HySIC構造C帯W級GaN増幅器の試作'', 電子情報通信学会和文論文誌C, vol.J104-C, no.2, pp.1--8, Feb. 2021. \\
% \bibitem{miso}
% \underline{D. Kobuchi}, Y. Narusue and H. Morikawa, ``Cancellation Conditions of Magnetic Field Leakage from Inductive Power Transfer Systems,'' IEEE Transactions on Vehicular Technology (under review).
\end{publication}


%----------------------------------------------------------------------
\section*{研究会}
\begin{publication}{99}
\bibitem{wpt}
\underline{大関啓史}, 小渕大輔, 成末義哲, 森川博之,
``磁界結合型無線給電のための磁気双極子モーメントに基づく汎用型漏洩磁界相殺機構の検討,\<''
電子情報通信学会 WPT研究会, April 2022. (発表予定) 


\end{publication}




%----------------------------------------------------------------------
\section*{国際コンテスト}
\begin{publication}{99}
%%-------書き方参考------%
% \bibitem{mfujishiro:apmc}
% M. Fujishiro, \underline{D. Kobuchi}, R. Kondo, Y. Narusue and H. Morikawa, 
% ``WPT Mini 4WD 400-cm Drag Race,\<''
% APMC2018 Student and Young Engineer Design Competitions, Kyoto, Japan, Nov. 2018. \\
% (First Prize)

\bibitem{itu5g2021}
Bojian Du, \underline{Hirofumi Ohzeki}, Ryoma Kondo and Daisuke Kobuchi, 
``ML5G-PHY-Reinforcement learning: scheduling and resource allocation,\<''
 ITU AI/ML in 5G Challenge, Dec. 2021.(Mentors encouragement award)
 
\bibitem{itu5g2020}
Ryoma Kondo, Takashi Ubukata, Kentaro Matsuura and \underline{Hirofumi Ohzeki},
``Route information failure in IP core networks,\<''
 ITU AI/ML in 5G Challenge, Dec. 2020. (the Grand Challenge Finale finalists)
 
\end{publication}


%%-------書き方参考------%
%----------------------------------------------------------------------
% \section*{研究会}
% \begin{publication}{99}

% \bibitem{poster2019}
% 松浦賢太郎, \underline{小渕大輔}, 成末義哲, 濱野皓志, 鈴木絢子, 吉田賢史, 西川健二郎, 森川博之, 川崎繁男, ``RFエナジーハーベスタによるワイヤレスセンサの現況,''
% 第19回宇宙科学シンポジウム, P-135, Jan. 2019.

% \bibitem{wpt}
% \underline{小渕大輔}, 成末義哲, 川原圭博, 森川博之,
% ``磁界結合型無線電力伝送における漏洩磁界低減型送電器アレイ駆動方式の検討,\<''
% 信学技報, vol. 119, no. 11, WPT2019-1, pp. 1 -- 6, April 2019. 

% \bibitem{ap}
% \underline{小渕大輔}, 松浦賢太郎, 成末義哲, 川崎繁男,
% ``モノトーン無線通信電力伝送に向けた C 帯 10 W 級薄型 アクティブ集積アレーアンテナ用 GaN アンプ,''
% 信学技報, vol. 119, no. 204, MW2019-62, pp. 37 -- 42, Sept 2019. 

% \bibitem{poster2020}
% \underline{小渕大輔}, 足立真志, 小原拓也, 清水駿斗, 米田崚平, 鈴木絢子, 松浦賢太郎, ``ワイヤレスセンサシステムの実現に向けた高周波デバイスの開発,''
% 第20回宇宙科学シンポジウム, P1-74, Jan. 2020.

% \bibitem{matsuura}
% 松浦賢太郎,\underline{小渕大輔},成末義哲,森川博之,
% ``磁界共振結合型無線電力伝送における自律的二次側共振周波数補正機構の検討,''
% 信学技報, vol. 120, no. 278, WPT2020-26, pp. 1 -- 6, Sept 2020. 

% \end{publication}

%----------------------------------------------------------------------
% \section*{全国大会}
% \begin{publication}{99}

% \bibitem{sou_2020_1}
% 薮田直人, 後藤優花, \underline{小渕大輔}, 内海淳, 中岡俊裕, 吉田賢史, 西川健二郎, 正光義則, 川崎繁男,
% ``宇宙情報通信エネルギー技術のためのRF HySICデバイスに関する研究,\<''
% 信学総大, C-2-22, March 2020.

% \bibitem{sou_2020_2}
% 吉田賢史, \underline{小渕大輔}, 松浦賢太郎, 西川健二郎, 川崎繁男,
% ``K帯無線通信とC帯マイクロ波無線電力伝送の両立によるワイヤレスセンサシステムの検討,\<''
% 信学総大, C-2-75, March 2020.

% \bibitem{soc_2020}
% \underline{小渕大輔}, 松浦賢太郎, 成末義哲, 森川博之,
% ``磁界結合型無線電力伝送における高調波磁界強度低減に向けた入力電圧設計,''
% 信学ソ大, B-20-2, Sept. 2020.

% \bibitem{sou_2021_1}
% \underline{小渕大輔}, 松浦賢太郎, 成末義哲, 森川博之,
% ``磁界結合型無線給電における高調波磁界低減型入力電圧波形設計の実験評価,''
% 信学総大, March. 2021.(to be appeared)


% \end{publication}

%----------------------------------------------------------------------


