%#!make thesis.dvi
%======================================================================
% 謝辞
%======================================================================
\chapter*{\thanksname}\label{chap:acknowledge}
\addcontentsline{toc}{chapter}{\thanksname}
\makeatletter\@mkboth{\thanksname}{}\makeatother

本研究は東京大学工学系研究科電気系工学専攻森川・成末研究室の森川博之教授,成末義哲准教授のご指導の下で行いました.本研究を進めるにあたり多くの方に支えて頂きました.ここに深く感謝の気持ちを表します.\\
森川博之教授には,本研究を行う機会を与えていただくと共に,オフィスアワーなどを通して多くの助言を賜りました.とりわけ研究に取り組む姿勢や未来に必要な技術を考え続けることの意義をご指導いただきました.深く感謝申し上げます.\\
成末義哲准教授には,本研究の基礎知識を学ばせていただくとともに,研究を進める上での考え方を数多く教えていただきました.研究テーマを考える段階から個別のミーティングを何度も組んでいただき、本当にありがとうございました.豊富な知識と経験に基づいた意見を下さり,多くの着眼点に気付かされ,また,日々の進捗に関して適切な助言をいただきました.心から感謝申し上げます.\\
修士課程の佐藤龍吾氏には,同じ通信グループの仲間としてテーマに関するアドバイスを近い視点からいただきました.深く感謝申し上げます.修士課程の嶋田大地氏には,研究に関する基礎事項を学ばせていただき,その細部まで探究する姿勢には驚かされました.深く感謝申し上げます.\\
博士課程の林政東氏,朱天昊氏には,大学院に入学し右も左も分からない私の質問に都度丁寧に答えてくださり,多くのサポートをいただきました.深く感謝いたします.博士課程の山口温大氏には,弛まず堅実に研究を進めていく姿勢から多くの刺激をいただきました.深く感謝いたします.\\
修士課程2年生の森永智大氏には,研究の合間の雑談を通じてリラックスする機会をいただきました.深く感謝いたします.修士課程2年生の杉村洋介氏には,常にポジティブな態度で研究に腰を据えて取り組む姿からモチベーションを高めていただきました.深く感謝いたします.修士課程2年生の加賀谷湧氏には,規則正しくストイックな研究姿勢を見せる一方でメリハリのついた生活を送る姿から,自分自身も頑張ろうという気持ちをいただきました.深く感謝いたします.修士課程2年生の范維翰氏には,研究生活で苦しいときにお互いに支え合うことができ,精神的な安定をいただきました.深く感謝いたします.\\
修士課程1年生の大久保諒一氏と蔡海濤氏には,入学当初から公私両面でお世話になり,楽しく知的な研究生活を送ることができました.深く感謝いたします.秘書の川北敦子さんには,快適に研究活動が行えるよう事務面でサポートしていただきました.深く感謝申し上げます.\\
学科同期の曹晨亭氏,全濟旭氏には,研究に関する情報交換や土日を使った研究活動に付き合っていただき,プライベートでも親しくしていただきました.深く感謝いたします.矢口真那斗氏には,AI関連の知識を多くご教示いただきました.深く感謝いたします.\\
最後に,私の研究活動を理解し,温かく寛容に支えてくれた家族に感謝の意を表し,謝辞の結びとします.\\

\textcolor{white}{追記:オードリーがパーソナリティを務めるラジオ「オールナイトニッポン」からは大学院生活における癒しと活力を日々賜りました.ここに心からの感謝の意を述べさせていただきます.}


\begin{flushright}
 \date{2025年1月22日}
\end{flushright}
