%#!make thesis.dvi
%======================================================================
% 謝辞
%======================================================================
\chapter*{\thanksname}\label{chap:acknowledge}
\addcontentsline{toc}{chapter}{\thanksname}
\makeatletter\@mkboth{\thanksname}{}\makeatother

本研究は東京大学工学系研究科電気系工学専攻森川・成末研究室の森川博之教授,成末義哲講師のご指導の下で行いました.本研究を進めるにあたり多くの方に支えて頂きました.ここに深く感謝の気持ちを表します.\\
森川博之教授には,本研究を行う機会を与えていただくと共に,全体ミーティングなどを通して多くの助言を賜りました.とりわけ自分の専門知識を他者に簡潔に的確に伝える難しさをご指導いただきました.深く感謝申し上げます.\\
成末義哲講師には,直接の指導者として本研究の基礎知識を学ばせていただくとともに,研究を進める上での考え方を数多く教えていただきました.豊富な知識と経験に基づいた意見を下さり,多くの着眼点に気付かされ,また,日々の進捗に関して適切な助言をいただきました.心から感謝申し上げます.\\
博士課程の生形貴氏には同じ通信グループの先輩としてテーマに関するアドバイスを近い視点から下さりました.深く感謝申し上げます.博士課程の杜博見氏は、研究生活の中でよく気にかけてくださり、息抜きの機会を多くいただきました。深く感謝いたします。博士課程の松浦賢太郎氏には、その弛まず堅実に研究を進めていく姿勢から刺激を多くいただきました。深く感謝いたします。博士課程の小渕大輔氏には、本研究内容に取り組むにあたり、親身に相談にのっていただき、中でも実験を行う際には多大な助力をいただきました。また研究室同期として度々雑談などを通して無線給電技術に関する知識を共有してくださいました。深く感謝いたします。博士課程の近藤亮磨氏には、その豊富な経験と広い見識から機械学習に関する知見やアドバイスを大変多くいただきました。深く感謝いたします。博士課程の関口浩氏は、いつも研究室で明るく話す姿が印象的で色々なお話を通して研究生活の活力をいただきました。深く感謝いたします。博士課程の小林司氏には、社会人博士という立場で自分にとっては全く新しい研究分野の知見を多く広げてくださいました。深く感謝申し上げます。博士課程のMeshal Almashan氏には、原稿の英語の添削の際に多大にご助力いただきました。感謝申し上げます。博士課程の佐竹孝氏には、社会人という視点から通信分野全体に関する意見をいただきました。感謝申し上げます。\\
修士2年生の足立真志氏には、同時期に修士論文を提出するにあたり、その研究に取り組む姿勢から活力をいただきました。深く感謝申し上げます。修士1年生の星智文氏, Uugantsetseg Tsolmon氏には、同じグループの後輩として,テーマを探す際の新しい研究分野に次々と着目していく姿から刺激をいただきました。深く感謝申し上げます。学部4年生の田島和樹氏,舘野将寿氏,澁谷舜氏,丁晟氏,アン ジェソル氏には、研究に取り組むにあたっての初心をいつも思い出させていただきました。深く感謝申し上げます。秘書の川北敦子さんには快適に研究活動が行えるように事務面のサポートをしていただきました.深く感謝申し上げます.\\
最後に私の研究活動を理解し,温かく寛容に支えてくれた家族に感謝の意を表し,謝辞の結びとします.\\

\textcolor{white}{追記:小原好美様がパーソナリティを務めるラジオ「小原好美のココロおきなく」からは大学院生活における癒しと活力を日々賜りました.ここに心からの感謝の意を述べさせていただきます.}


\begin{flushright}
 \date{2022年1月27日}
\end{flushright}
