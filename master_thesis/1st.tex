%#!make thesis.dvi
%======================================================================
% 序論
%======================================================================
\chapter{序論}{}
\label{chap:1st}

\section{本論文の背景と目的}

% ステップ①:社会的・技術的な大背景の提示
5G以降の移動体通信では,高周波数帯の利用と大規模MIMO技術の導入が進んでいる.
基地局は多数のアンテナ素子を用いてビームフォーミングやプリコーディングを行い,空間多重により通信容量を確保する.
これらの送信処理では,送受信間の伝搬特性を表すチャネル状態情報が不可欠である.
チャネル状態情報の誤差はビームの指向ずれや干渉増大を招き,通信品質の劣化に直結する.

チャネル状態情報は,端末が送信する参照信号に基づいて基地局が推定する.
アンテナ素子数や同時接続ユーザ数が増加すると,チャネル状態情報の次元と更新頻度の要求が増大する.
参照信号の送信頻度を高めれば推定精度は向上するが,時間周波数リソースと送信電力を消費し,データ伝送に割ける資源を圧迫する.
加えて,移動や遮蔽物によりチャネルが時間変動する環境では,推定から利用までの遅延によりチャネル状態情報が陳腐化するチャネルエイジングが問題となる.
以上より,チャネル状態情報の推定精度と更新コストの両立が課題となっている.

% ステップ②:解決策(キー技術)の導入とメリット
参照信号のオーバーヘッドを抑えつつ将来のチャネル状態情報を推定する手段として,時系列CSI予測が研究されている.
時系列CSI予測は,過去の観測系列から将来時刻のチャネル状態情報を外挿する枠組みである.
予測値をビームフォーミングやプリコーディングに用いれば,参照信号の送信頻度を低減しながら通信品質を維持できる.
近年は深層学習を用いてチャネルの時間変動をデータから学習する手法が提案されており,古典的なウィナーフィルタやカルマンフィルタに比べて柔軟なモデル化が期待できる.

% ステップ③:研究課題の具体化
時系列CSI予測を実環境で運用するには,学習時と異なる環境への適応が求められる.
基地局配置,周波数帯,散乱特性,移動速度が変化すると,訓練分布とテスト分布が不一致となり,予測誤差が増大する.
セル内の建造物の増減や時間帯に伴うユーザ密度の変化も,電波伝搬環境を変化させる要因となる.
このような分布変化に追従するには,運用中にモデルを更新する仕組みが必要である.

既存のモデル更新手法の多くは,新環境で得られる正解データを用いた教師あり学習を前提としている.
一方,時系列CSI予測で参照信号を削減する場合,予測スロットでは参照信号を送信しないため,予測対象時刻のチャネル状態情報を観測できない.
正解ラベルが欠損する状況では,予測値と真値の誤差を計算できず,教師あり学習に基づくモデル更新が困難となる.
予測を一時停止して参照信号を送信すれば正解ラベルを取得できるが,参照信号削減という時系列CSI予測の目的に反する.
以上より,参照信号削減を維持しながら予測モデルを更新する枠組みが必要である.

% ステップ④:本研究の目的と貢献(提案手法)
本研究では,参照信号削減下において予測を継続しながら時系列CSI予測モデルを更新する機構を提案する.
提案機構の要点は,欠損した正解ラベルを補間により事後的に推定し,補間値を含む系列から更新用の学習サンプルを構成する点にある.
予測スロットより後に得られる観測値を用いて補間値を算出するため,参照信号削減を維持したまま学習サンプルを生成できる.
本論文では,複数の補間手法を比較して有効な手法を選定し,提案機構の有効性をシミュレーションにより検証する.
評価シナリオとして,新規基地局への展開とセル内の環境変動を想定し,いずれの場合においても予測精度が改善することを示す.


% NOTE: 参照先画像が存在しないためコメントアウト(ビルドエラー回避)
% \begin{figure}[htbp]
%   \centering
%   \includegraphics[width=0.8\linewidth]{img/2nd/test.png.pdf}
%   \caption{テスト画像の挿入}
%   \label{fig:test_image}
% \end{figure}

\section{本論文の構成}

\noindent
本論文は全5章で構成される.
第1章では,本研究の背景と目的を述べる.
第2章では,5G NR物理層における無線伝搬とCSI取得の枠組みを整理し,時系列CSI予測が必要となる背景と関連研究,および環境適応における課題を述べる.
第3章では,参照信号削減に伴い欠損するCSIを扱うための時系列補間手法を定式化し,レイトレーシングにより生成したCSI系列を用いて補間精度を比較する.
第4章では,補間値に基づき更新用サンプルを構成して予測モデルを更新する機構を提案し,複数のシナリオで有効性を評価する.
第5章では,本研究で得られた知見をまとめ,今後の課題を述べる.